\documentclass[12pt,titlepage]{article}

\usepackage{amsmath}
\usepackage{mathrsfs}
\usepackage{amsfonts}
\usepackage{amssymb}
\usepackage{amsthm}
\usepackage{mathtools}
\usepackage{graphicx}
\usepackage{color}
\usepackage{ucs}
\usepackage[utf8x]{inputenc}
\usepackage{xparse}
\usepackage{hyperref}

%----Macros----------
%
% Unresolved issues:
%
%  \righttoleftarrow
%  \lefttorightarrow
%
%  \color{} with HTML colorspec
%  \bgcolor
%  \array with options (without options, it's equivalent to the matrix environment)

% Of the standard HTML named colors, white, black, red, green, blue and yellow
% are predefined in the color package. Here are the rest.
\definecolor{aqua}{rgb}{0, 1.0, 1.0}
\definecolor{fuschia}{rgb}{1.0, 0, 1.0}
\definecolor{gray}{rgb}{0.502, 0.502, 0.502}
\definecolor{lime}{rgb}{0, 1.0, 0}
\definecolor{maroon}{rgb}{0.502, 0, 0}
\definecolor{navy}{rgb}{0, 0, 0.502}
\definecolor{olive}{rgb}{0.502, 0.502, 0}
\definecolor{purple}{rgb}{0.502, 0, 0.502}
\definecolor{silver}{rgb}{0.753, 0.753, 0.753}
\definecolor{teal}{rgb}{0, 0.502, 0.502}

% Because of conflicts, \space and \mathop are converted to
% \itexspace and \operatorname during preprocessing.

% itex: \space{ht}{dp}{wd}
%
% Height and baseline depth measurements are in units of tenths of an ex while
% the width is measured in tenths of an em.
\makeatletter
\newdimen\itex@wd%
\newdimen\itex@dp%
\newdimen\itex@thd%
\def\itexspace#1#2#3{\itex@wd=#3em%
\itex@wd=0.1\itex@wd%
\itex@dp=#2ex%
\itex@dp=0.1\itex@dp%
\itex@thd=#1ex%
\itex@thd=0.1\itex@thd%
\advance\itex@thd\the\itex@dp%
\makebox[\the\itex@wd]{\rule[-\the\itex@dp]{0cm}{\the\itex@thd}}}
\makeatother

% \tensor and \multiscript
\makeatletter
\newif\if@sup
\newtoks\@sups
\def\append@sup#1{\edef\act{\noexpand\@sups={\the\@sups #1}}\act}%
\def\reset@sup{\@supfalse\@sups={}}%
\def\mk@scripts#1#2{\if #2/ \if@sup ^{\the\@sups}\fi \else%
  \ifx #1_ \if@sup ^{\the\@sups}\reset@sup \fi {}_{#2}%
  \else \append@sup#2 \@suptrue \fi%
  \expandafter\mk@scripts\fi}
\def\tensor#1#2{\reset@sup#1\mk@scripts#2_/}
\def\multiscripts#1#2#3{\reset@sup{}\mk@scripts#1_/#2%
  \reset@sup\mk@scripts#3_/}
\makeatother

% \slash
\makeatletter
\newbox\slashbox \setbox\slashbox=\hbox{$/$}
\def\itex@pslash#1{\setbox\@tempboxa=\hbox{$#1$}
  \@tempdima=0.5\wd\slashbox \advance\@tempdima 0.5\wd\@tempboxa
  \copy\slashbox \kern-\@tempdima \box\@tempboxa}
\def\slash{\protect\itex@pslash}
\makeatother

% math-mode versions of \rlap, etc
% from Alexander Perlis, "A complement to \smash, \llap, and lap"
%   http://math.arizona.edu/~aprl/publications/mathclap/
\def\clap#1{\hbox to 0pt{\hss#1\hss}}
\def\mathllap{\mathpalette\mathllapinternal}
\def\mathrlap{\mathpalette\mathrlapinternal}
\def\mathclap{\mathpalette\mathclapinternal}
\def\mathllapinternal#1#2{\llap{$\mathsurround=0pt#1{#2}$}}
\def\mathrlapinternal#1#2{\rlap{$\mathsurround=0pt#1{#2}$}}
\def\mathclapinternal#1#2{\clap{$\mathsurround=0pt#1{#2}$}}

% Renames \sqrt as \oldsqrt and redefine root to result in \sqrt[#1]{#2}
\let\oldroot\root
\def\root#1#2{\oldroot #1 \of{#2}}
\renewcommand{\sqrt}[2][]{\oldroot #1 \of{#2}}

% Manually declare the txfonts symbolsC font
\DeclareSymbolFont{symbolsC}{U}{txsyc}{m}{n}
\SetSymbolFont{symbolsC}{bold}{U}{txsyc}{bx}{n}
\DeclareFontSubstitution{U}{txsyc}{m}{n}

% Manually declare the stmaryrd font
\DeclareSymbolFont{stmry}{U}{stmry}{m}{n}
\SetSymbolFont{stmry}{bold}{U}{stmry}{b}{n}

% Manually declare the MnSymbolE font
\DeclareFontFamily{OMX}{MnSymbolE}{}
\DeclareSymbolFont{mnomx}{OMX}{MnSymbolE}{m}{n}
\SetSymbolFont{mnomx}{bold}{OMX}{MnSymbolE}{b}{n}
\DeclareFontShape{OMX}{MnSymbolE}{m}{n}{
    <-6>  MnSymbolE5
   <6-7>  MnSymbolE6
   <7-8>  MnSymbolE7
   <8-9>  MnSymbolE8
   <9-10> MnSymbolE9
  <10-12> MnSymbolE10
  <12->   MnSymbolE12}{}

% Declare specific arrows from txfonts without loading the full package
\makeatletter
\def\re@DeclareMathSymbol#1#2#3#4{%
    \let#1=\undefined
    \DeclareMathSymbol{#1}{#2}{#3}{#4}}
\re@DeclareMathSymbol{\neArrow}{\mathrel}{symbolsC}{116}
\re@DeclareMathSymbol{\neArr}{\mathrel}{symbolsC}{116}
\re@DeclareMathSymbol{\seArrow}{\mathrel}{symbolsC}{117}
\re@DeclareMathSymbol{\seArr}{\mathrel}{symbolsC}{117}
\re@DeclareMathSymbol{\nwArrow}{\mathrel}{symbolsC}{118}
\re@DeclareMathSymbol{\nwArr}{\mathrel}{symbolsC}{118}
\re@DeclareMathSymbol{\swArrow}{\mathrel}{symbolsC}{119}
\re@DeclareMathSymbol{\swArr}{\mathrel}{symbolsC}{119}
\re@DeclareMathSymbol{\nequiv}{\mathrel}{symbolsC}{46}
\re@DeclareMathSymbol{\Perp}{\mathrel}{symbolsC}{121}
\re@DeclareMathSymbol{\Vbar}{\mathrel}{symbolsC}{121}
\re@DeclareMathSymbol{\sslash}{\mathrel}{stmry}{12}
\re@DeclareMathSymbol{\bigsqcap}{\mathop}{stmry}{"64}
\re@DeclareMathSymbol{\biginterleave}{\mathop}{stmry}{"6}
\re@DeclareMathSymbol{\invamp}{\mathrel}{symbolsC}{77}
\re@DeclareMathSymbol{\parr}{\mathrel}{symbolsC}{77}
\makeatother

% \llangle, \rrangle, \lmoustache and \rmoustache from MnSymbolE
\makeatletter
\def\Decl@Mn@Delim#1#2#3#4{%
  \if\relax\noexpand#1%
    \let#1\undefined
  \fi
  \DeclareMathDelimiter{#1}{#2}{#3}{#4}{#3}{#4}}
\def\Decl@Mn@Open#1#2#3{\Decl@Mn@Delim{#1}{\mathopen}{#2}{#3}}
\def\Decl@Mn@Close#1#2#3{\Decl@Mn@Delim{#1}{\mathclose}{#2}{#3}}
\Decl@Mn@Open{\llangle}{mnomx}{'164}
\Decl@Mn@Close{\rrangle}{mnomx}{'171}
\Decl@Mn@Open{\lmoustache}{mnomx}{'245}
\Decl@Mn@Close{\rmoustache}{mnomx}{'244}
\Decl@Mn@Open{\llbracket}{stmry}{'112}
\Decl@Mn@Close{\rrbracket}{stmry}{'113}
\makeatother

% Widecheck
\makeatletter
\DeclareRobustCommand\widecheck[1]{{\mathpalette\@widecheck{#1}}}
\def\@widecheck#1#2{%
    \setbox\z@\hbox{\m@th$#1#2$}%
    \setbox\tw@\hbox{\m@th$#1%
       \widehat{%
          \vrule\@width\z@\@height\ht\z@
          \vrule\@height\z@\@width\wd\z@}$}%
    \dp\tw@-\ht\z@
    \@tempdima\ht\z@ \advance\@tempdima2\ht\tw@ \divide\@tempdima\thr@@
    \setbox\tw@\hbox{%
       \raise\@tempdima\hbox{\scalebox{1}[-1]{\lower\@tempdima\box
\tw@}}}%
    {\ooalign{\box\tw@ \cr \box\z@}}}
\makeatother

% \mathraisebox{voffset}[height][depth]{something}
\makeatletter
\NewDocumentCommand\mathraisebox{moom}{%
\IfNoValueTF{#2}{\def\@temp##1##2{\raisebox{#1}{$\m@th##1##2$}}}{%
\IfNoValueTF{#3}{\def\@temp##1##2{\raisebox{#1}[#2]{$\m@th##1##2$}}%
}{\def\@temp##1##2{\raisebox{#1}[#2][#3]{$\m@th##1##2$}}}}%
\mathpalette\@temp{#4}}
\makeatletter

% udots (taken from yhmath)
\makeatletter
\def\udots{\mathinner{\mkern2mu\raise\p@\hbox{.}
\mkern2mu\raise4\p@\hbox{.}\mkern1mu
\raise7\p@\vbox{\kern7\p@\hbox{.}}\mkern1mu}}
\makeatother

%% Fix array
\newcommand{\itexarray}[1]{\begin{matrix}#1\end{matrix}}
%% \itexnum is a noop
\newcommand{\itexnum}[1]{#1}

%% Renaming existing commands
\newcommand{\underoverset}[3]{\underset{#1}{\overset{#2}{#3}}}
\newcommand{\widevec}{\overrightarrow}
\newcommand{\darr}{\downarrow}
\newcommand{\nearr}{\nearrow}
\newcommand{\nwarr}{\nwarrow}
\newcommand{\searr}{\searrow}
\newcommand{\swarr}{\swarrow}
\newcommand{\curvearrowbotright}{\curvearrowright}
\newcommand{\uparr}{\uparrow}
\newcommand{\downuparrow}{\updownarrow}
\newcommand{\duparr}{\updownarrow}
\newcommand{\updarr}{\updownarrow}
\newcommand{\gt}{>}
\newcommand{\lt}{<}
\newcommand{\map}{\mapsto}
\newcommand{\embedsin}{\hookrightarrow}
\newcommand{\Alpha}{A}
\newcommand{\Beta}{B}
\newcommand{\Zeta}{Z}
\newcommand{\Eta}{H}
\newcommand{\Iota}{I}
\newcommand{\Kappa}{K}
\newcommand{\Mu}{M}
\newcommand{\Nu}{N}
\newcommand{\Rho}{P}
\newcommand{\Tau}{T}
\newcommand{\Upsi}{\Upsilon}
\newcommand{\omicron}{o}
\newcommand{\lang}{\langle}
\newcommand{\rang}{\rangle}
\newcommand{\Union}{\bigcup}
\newcommand{\Intersection}{\bigcap}
\newcommand{\Oplus}{\bigoplus}
\newcommand{\Otimes}{\bigotimes}
\newcommand{\Wedge}{\bigwedge}
\newcommand{\Vee}{\bigvee}
\newcommand{\coproduct}{\coprod}
\newcommand{\product}{\prod}
\newcommand{\closure}{\overline}
\newcommand{\integral}{\int}
\newcommand{\doubleintegral}{\iint}
\newcommand{\tripleintegral}{\iiint}
\newcommand{\quadrupleintegral}{\iiiint}
\newcommand{\conint}{\oint}
\newcommand{\contourintegral}{\oint}
\newcommand{\infinity}{\infty}
\newcommand{\bottom}{\bot}
\newcommand{\minusb}{\boxminus}
\newcommand{\plusb}{\boxplus}
\newcommand{\timesb}{\boxtimes}
\newcommand{\intersection}{\cap}
\newcommand{\union}{\cup}
\newcommand{\Del}{\nabla}
\newcommand{\odash}{\circleddash}
\newcommand{\negspace}{\!}
\newcommand{\widebar}{\overline}
\newcommand{\textsize}{\normalsize}
\renewcommand{\scriptsize}{\scriptstyle}
\newcommand{\scriptscriptsize}{\scriptscriptstyle}
\newcommand{\mathfr}{\mathfrak}
\newcommand{\statusline}[2]{#2}
\newcommand{\tooltip}[2]{#2}
\newcommand{\toggle}[2]{#2}

% Theorem Environments
\theoremstyle{plain}
\newtheorem{theorem}{Theorem}
\newtheorem{lemma}{Lemma}
\newtheorem{prop}{Proposition}
\newtheorem{cor}{Corollary}
\newtheorem*{utheorem}{Theorem}
\newtheorem*{ulemma}{Lemma}
\newtheorem*{uprop}{Proposition}
\newtheorem*{ucor}{Corollary}
\theoremstyle{definition}
\newtheorem{defn}{Definition}
\newtheorem{example}{Example}
\newtheorem*{udefn}{Definition}
\newtheorem*{uexample}{Example}
\theoremstyle{remark}
\newtheorem{remark}{Remark}
\newtheorem{note}{Note}
\newtheorem*{uremark}{Remark}
\newtheorem*{unote}{Note}

%-------------------------------------------------------------------

\begin{document}

%-------------------------------------------------------------------

\section*{Linear Temporal Logic and Queueing}

This page discusses the paper by \href{http://sites.bu.edu/hyness/files/2016/02/coogan2015trafficltl.pdf}{Coogan, et al} and also has some theoretical background for different queueing disciplines.

\hypertarget{queueing_disciplines}{}\subsubsection*{{Queueing Disciplines}}\label{queueing_disciplines}

There are many different disciplines for queueing. The FIFO (first-in-first-out) is the general standard. We also have several others worth discussing, and mathematical ways to represent them.

The notation described below is from \href{https://dl.acm.org/doi/10.1145/322003.322009}{Chandy, et al}. The queueing disciplines I'll list here are FIFO, LCFSPR (last-come-first-served preemptive resume), PS (processor sharing), IS (infinite server), preemptive fixed priority, and nonpreemptive fixed priority.

In particular, we will focus on expressing the queueing disciplines with two families of parameters: the arrival probabilities and the service rates.

\emph{Customer classes: There are $K$ different classes of customers who use the queue, numbered $k = 1,2,\ldots K$. Each class has an arrival rate $\lambda_{k}$ according to a Poisson process, and a service distribution $F_{k}$, which is general, though usually taken to be exponential.}

\emph{States: Chandy et al view the queue as a set of stations, each of which is occupied by one customer. If there are $n$ customers in the queue, then the occupied stations are indexed by $i=1,2,\ldots n$. $k(i)$ is the class of the customer at station $i$. The \textbf{state} of the queue is given by $S = (k(1),\ldots,k(n))$and gives the occupancy of the queue. An example state is}

\begin{displaymath}
S = (3,2,1,1,2,2)
\end{displaymath}
which shows that we have 6 customers in queue, the first of which is class 3, the second of class 2, etc.

\emph{Remark: The complete state of the queue needs to also account for the remaining service requirement of station $i$, but we don't need that stuff right now.}

\begin{itemize}%
\item Queueing discipline: Here is where we describe the queueing disciplines. They are described using the conditional arrival probabilities $a(i|S,k)$ which describes the probability an arriving customer will enter station $i$, given the current state $S$ and the customer's class $k$; and the service rates $r(i|S)$, which describes the service rate for the customer at station $i$, given the occupancy state $S$. The \textbf{gross service rate} $R(S) = \sum_{i=1}^{n}r(i|S)$, and is typically 1.

\end{itemize}
We now characterize the above service disciplines and explain:

\begin{enumerate}%
\item FCFS: First-Come-First-Serve. We have that

\begin{displaymath}
a(i|S,k) = \delta_{i,n+1}
\end{displaymath}
where $\delta_{i,n+1} = 1$ if $i=n+1$, and 0 otherwise. This means that an arriving customer goes to the back of the line with probability 1 and nowhere else, regardless of class.

\begin{displaymath}
r(i|S) = \delta_{i,1}
\end{displaymath}
Here this means that the server is serving only the first customer in line.


\item LCFSPR: Last-Come-First-Serve-Preemptive-Resume

\begin{displaymath}
a(i|S,k) = \delta_{i,1}
\end{displaymath}
This time, the arriving customer is put at the top of the stack as soon as it arrives, regardless of class, and nowhere else.

\begin{displaymath}
r(i|S) = \delta_{i,1}
\end{displaymath}
As with FCFS, the first customer in the queue is served, and only this one.


\item PS: Processor Sharing

\begin{displaymath}
a(i|S,k) = \frac{1}{n+1}
\end{displaymath}
Here, the arriving customer (which makes a total of $n+1$ customers) can go anywhere in slots $i=1,\ldots n+1$ with equal probaility.

\begin{displaymath}
r(i|S) = \frac{1}{n}
\end{displaymath}
Here, the server is shared equally among all current customers. The rate each customer gets is the $1/n$ fraction of the rate a lone customer would get.


\item IS: Infinite Server

\begin{displaymath}
a(i|S,k) = \frac{1}{n+1}
\end{displaymath}
Same as PS.

\begin{displaymath}
r(i|S) = 1
\end{displaymath}
Here, we assume that the server has the capacity to give its full attention to every single customer in the queue at the same time. Basically ``perfect multitasking''. This is mostly a theoretical limit and not used in practice often.


\item PFP: Preemptive Fixed Priority

\begin{displaymath}
a(i|S,k) = \delta_{i,1+\sum_{j=1}^{k}n_{j}}
\end{displaymath}
Here we'll have to expand. With fixed priority queueing disciplines, we have that all customers of class 1 come before all customers of class 2, etc. So an arriving customer goes to the back of the line for its class. $n_{j}$ denotes the number of customers of class $j$ in the state $S$. For example, if $S = (1,1,2,3,3)$, and an arriving customer is of class 2, $a(i|S,k)$ is only 1 for $i = 1+(2+1) = 4$, which puts it in the 4th position behind the current class 2 customer. Preemptive fixed priority states that a customer of class 1 can push down a current customer of a lower class, so if $S = (2,2,3)$, and a customer of class 1 arrives, it will go to the head of the line and pre-empt the current customer who is class 2.

\begin{displaymath}
r(i|s) = \delta_{i,1}
\end{displaymath}
The server concentrates all its attention on the first customer in line.


\item NFP: Non-preemptive Fixed Priority This discipline is similar, but doesn't ``push down'' a customer of a lower class.

\begin{displaymath}
a(i|S,k) = \delta_{i, 2+\sum_{j=1}^{k}n_{j}}
\end{displaymath}
This shows us that if we have $S = (2,2,3)$, and a customer of class 1 arrives, it will go in position $2 + (0) = 2$, which is right behind the current class 2 customer, so the current head of the line isn't pushed away.

\begin{displaymath}
r(i|S) = \delta_{i,1}
\end{displaymath}


\end{enumerate}
\hypertarget{more_on_states}{}\subsubsection*{{More on States}}\label{more_on_states}

This section details a bit more about the states of the queue.

\hypertarget{feasibility}{}\paragraph*{{Feasibility}}\label{feasibility}

We call a state $S = (k(1),\ldots,k(n))$ \textbf{feasible} if

\begin{itemize}%
\item $S = \emptyset$, which corresponds to an empty queue, or
\item $S = S' + (i,k)$, where we obtain a new state from a feasible state by inserting a customer of class $k$ in station $i$, such that the arrival probability $a(i|S',k) \neq 0$, and is thus governed by the queueing discipline. For $S' = (k'(1),k'(2),\ldots, k'(n))$,\begin{displaymath}
S = S' + (i,k) = (k'(1),k'(2),\ldots,k'(i-1),k,k'(i+1),\ldots,k'(n))
\end{displaymath}
where we insert the new customer into slot $i$, pushing the customer formerly in slot $i$ to slot $i+1$, and so forth. The new length of the state is $n+1$. For example, for $S' = (2,3,3,1)$ that is presumed feasible, and an arrival customer $(2,1)$, then

\begin{displaymath}
S = S' + (2,1) = (2,3,3,1) + (2,1) = (2,1,3,3,1)
\end{displaymath}
where the arriving customer of class one is inserted into station 2, and is of class 1.


\item $S = S'-i$, where we remove the customer at station $i$ from a feasible state $S'$, where $r(i|S') \gt 0$. This is also governed by the queueing discipline.

\end{itemize}
We'll look at creating the set of feasible states with a few examples of queueing disciplines from before.

\begin{enumerate}%
\item FCFS: For FCFS, $a(i|S,k) = \delta_{i,n+1}$. The first feasible state is $\emptyset$. From there, we build $S = (k(1))$, which is a single customer in the queue with class $k(1)$. Following this, we can only add customers to the back of the line, regardless of the customer class. Thus, we can only get additional states of the form $S' + (i,k)$ where $i = n+1$. Via the subtraction/removal method, $r(i|S) = \delta_{i,1}$, and thus we can only remove customers from the first slot. Thus $S-1$ are the only feasible ``subtraction'' states.


\item PS: For Processor sharing, $a(i|S,k) = \frac{1}{n+1}$, meaning we are allowed to insert any customer of any class into any station $i \in \{1,2,\ldots, n+1\}$, and it's done with equal probability. $r(i|S) = \frac{1}{n}$, so removal from any station $i=1,\ldots,n$ is feasible if there is a customer in that position to remove. Then from $\emptyset$, we can build $S = (k(1))$. From there, we may insert the next customer either in front of the first, or behind, or we may remove that customer. Similarly, for $S = (k(1),\ldots,k(n))$, any $S + (i,k)$ is feasible for $i=1,\ldots n+1$, and any $S-i$ is feasible for $i=1,\ldots,n$.


\item PFP: Pre-emptive fixed priority has $a(i|S,k) = \delta_{i,1+\sum_{j=1}^{k}n_{j}}$, and $r(i|S) = \delta_{i,1}$. Then the set of feasible states gets a little more complicated. From $\emptyset$, we can build $S = (k(1))$. After that, what is feasible depends on $k(1)$. If the next arrival is a higher class, we must insert it ahead of the original customer. If the next arrival is the same class or lower, we must insert it behind the original. And like FCFS, we can only remove a customer from the head of the line, so only $S-1$ is feasible for feasible states $S$.



\end{enumerate}
We assign a probability of 0 to infeasible states and disregard them.

\hypertarget{complete_state_with_service_requirement}{}\paragraph*{{Complete State (with Service Requirement)}}\label{complete_state_with_service_requirement}

Regardless of situation, every customer comes with a service requirement that is random (sometimes called a service distribution). In the simplest case, the service requirement is given an exponential distribution for all customers, perhaps varying the rate with the class of customer. The exponential distribution is the most common one employed, because the distribution is memoryless, and we don't need to keep track of the residual work remaining while a customer is in service.

However, there are many cases where we want to allow something besides the exponential distribution as a service requirement. Since the exponential distribution is the only continuous distribution with the memoryless property, changing the service requirement to anything else will require us to keep track of the residual work for each customer. There are many different ways to handle this, and this can be a topic of discussion if necessary.

Where this is relevant to the state of the queue is to note that the state $S$ above only gives the discrete part, or occupancy, of the queue. We also need to tack on another vector that tracks each customer's remaining service requirement. The \textbf{complete state} $X = (S, X_{1},\ldots,X_{n})$ takes the occupancy $S$ and appends $X_{i}$, a variable giving the remaining service requirement. The service requirement is presume identical for each class of customers and is denoted $F_{k}$, $k=1,\ldots,K$. When a new customer of class $k$ is inserted at station $i$, then $X_{i}$ is chosen according to the distribution $F_{k}$. As a customer is served, his residual service requirement decreases continuously to 0, at which point the customer departs. $X_{i}$ decreases according to the rate $r(i|S)$ determined by the service discipline, so $\frac{d X_{i}}{dt} = -r(i|S)$.




\end{document}
